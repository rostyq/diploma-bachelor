\likechapter{ВСТУП}

Більшість поверхонь сучасних машин піддаються зносу. Як приклад можна навести, зуби шестерень редуктора, направляючі і повзуни, шийки колінчатих валів двигунів, внутрішня поверхня циліндрів двигунів внутрішнього згоряння і тому подібне. Тому до поверхонь деталей можуть висувати ряд вимог, які задають певні механічні властивості: міцність, зносостійкість, корозійна стійкість, жаростійкість, припрацьованість. Існують різні способи поверхневого зміцнення деталей, підвищення їх корозійної стійкості і зниження або збільшення тертя сполучених поверхонь. До них відносяться: поверхнева термічна обробка, легування поверхні деталі наплавленням сплавів, що відповідають необхідним вимогам, гальванічне нанесення на поверхню деталі антикорозійного покриття та інші.

Для зміцнення та нанесення захисних поверхонь вигідно використовувати електрофізичні методи обробки матеріалів, сутність яких полягає на використанні концентрованих потоків енергії. Одним із таких методів є електроіскрове легування (ЕІЛ) металевих поверхонь [1]. Його основними перевагами є економічність, екологічність та простота відтворення. Також цей метод дає широкий спектр можливих варіантів проведення експерименту в залежності від обраних матеріалів та відповідно отриманих властивостей поверхні. Саме по цим критеріям і обраний метод досліження на виробничій практиці.

В даному звіті розглянуті фізичні основи та відомі моделі проходження фізичного процесу електроіскрового легування. Відповідно у роботі наведені методика, результати експериментального дослідження отримання зносостійких поверхонь, висновки на основі літературного огляду та отриманих даних.
