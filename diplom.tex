 \documentclass[a4paper,14pt,ukrainian]{extreport}

%\usepackage{extsizes}
%\usepackage{cmap} % для кодировки шрифтов в pdf
%\usepackage[T2A]{fontenc}
%\usepackage[utf8]{inputenc}
\usepackage[ukrainian,english]{babel}


\usepackage{fontspec} % loaded by polyglossia, but included here for transparency 
\usepackage{polyglossia}
\setmainlanguage{ukrainian}
\setotherlanguage{english}
\newfontfamily\ukrainianfont[Script=Cyrillic]{Times New Roman}

\usepackage{kantlipsum} % для водички

\usepackage{graphicx} % для вставки картинок
\usepackage{amssymb,amsfonts,amsmath,amsthm} % математические дополнения от АМС
\usepackage{indentfirst} % отделять первую строку раздела абзацным отступом тоже
\usepackage[usenames,dvipsnames]{color} % названия цветов
\usepackage{makecell}
\usepackage{multirow} % улучшенное форматирование таблиц
%\usepackage{ulem} % подчеркивания

%%%%%%%%%%%%%%%%%%%%%%%%%% 
% КОЛОНТИТУЛЫ
%%%%%%%%%%%%%%%%%%%%%%%%%% 
\usepackage{fancyhdr}
\pagestyle{fancy}
\fancyhf{}
\fancyhead[R]{\thepage}
\fancyheadoffset{0mm}
\fancyfootoffset{0mm}
\setlength{\headheight}{17pt}
\renewcommand{\headrulewidth}{0pt}
\renewcommand{\footrulewidth}{0pt}
\fancypagestyle{plain}{ 
  \fancyhf{}
  \rhead{\thepage}}
\setcounter{page}{5} % начать нумерацию страниц с №5

%%%%%%%%%%%%%%%%%%%%%%%%%% 
% ПОДПИСИ К РИСУНКАМ
%%%%%%%%%%%%%%%%%%%%%%%%%% 
\usepackage[tableposition=top]{caption}
\usepackage{subcaption}
\DeclareCaptionLabelFormat{gostfigure}{Рисунок #2}
\DeclareCaptionLabelFormat{gosttable}{Таблиця #2}
\DeclareCaptionLabelSeparator{gost}{~—~}
\captionsetup{labelsep=gost}
\captionsetup{%
  format=plain,
  margin={1.25cm,0cm},
  indention=-1.25cm,
  font=doublespacing
}
\captionsetup[figure]{labelformat=gostfigure}
\captionsetup[table]{labelformat=gosttable}
\renewcommand{\thesubfigure}{\asbuk{subfigure}}

%%%%%%%%%%%%%%%%%%%%%%%%%% 
% МЕЛОЧИ
%%%%%%%%%%%%%%%%%%%%%%%%%% 
\parindent=1.25cm
\usepackage[doublespacing]{setspace}
%\usepackage[nodisplayskipstretch]{setspace}

%%%%%%%%%%%%%%%%%%%%%%%%%% 
% ОГЛАВЛЕНИЯ
%%%%%%%%%%%%%%%%%%%%%%%%%% 
\usepackage{titlesec}

\titleformat{\chapter}[display]
{\filcenter}
{\MakeUppercase{\chaptertitlename} \thechapter}
{8pt}
{\bfseries}{}

\titleformat{\section}[block]
{\normalsize\bfseries}
{\thesection}
{1em}{}

\titleformat{\subsection}[block]
{\normalsize\bfseries}
{\thesubsection}
{1em}{}

% Настройка вертикальных и горизонтальных отступов
\titlespacing*{\chapter}{0pt}{-30pt}{8pt}
\titlespacing*{\section}{\parindent}{*4}{*4}
\titlespacing*{\subsection}{\parindent}{*4}{*4}

%%%%%%%%%%%%%%%%%%%%%%%%%% 
% LIKECHAPTER
%%%%%%%%%%%%%%%%%%%%%%%%%% 
\newcommand{\empline}{\mbox{}\newline}
\newcommand{\likechapterheading}[1]{ 
  \begin{center}
    \newpage
    \textbf{\MakeUppercase{#1}}
  \end{center}
  \empline}

\makeatletter
\renewcommand{\@dotsep}{2}
\newcommand{\l@likechapter}[2]{{\bfseries\@dottedtocline{0}{0pt}{0pt}{#1}{#2}}}
\makeatother
\newcommand{\likechapter}[1]{    
  \likechapterheading{#1}    
  \addcontentsline{toc}{likechapter}{\MakeUppercase{#1}}}

%%%%%%%%%%%%%%%%%%%%%%%%%% 
% ПРИЛОЖЕНИЯ
%%%%%%%%%%%%%%%%%%%%%%%%%% 
\usepackage[title,titletoc]{appendix}

\titleformat{\paragraph}[display]
{\filcenter}
{\MakeUppercase{\chaptertitlename} \thechapter}
{8pt}
{\bfseries}{}
\titlespacing*{\paragraph}{0pt}{-30pt}{8pt}

\newcommand{\append}[1]{  
  \clearpage
  \stepcounter{chapter}    
  \paragraph{\MakeUppercase{#1}}
  \empline
  \addcontentsline{toc}{likechapter}{\MakeUppercase{\chaptertitlename~\Asbuk{chapter}\;#1}}}

%%%%%%%%%%%%%%%%%%%%%%%%%% 
% ЗМІСТ
%%%%%%%%%%%%%%%%%%%%%%%%%% 
\usepackage{tocloft}
\renewcommand{\cfttoctitlefont}{\hspace{0.38\textwidth} \bfseries\MakeUppercase}
\renewcommand{\cftbeforetoctitleskip}{-1em}
\renewcommand{\cftaftertoctitle}{\mbox{}\hfill \\ \mbox{}\hfill{\footnotesize Стор.}\vspace{-2.5em}}
\renewcommand{\cftchapfont}{\normalsize\bfseries \MakeUppercase{\chaptername} }
\renewcommand{\cftsecfont}{\hspace{48pt}} % было 31pt
\renewcommand{\cftsubsecfont}{\hspace{16.5pt}} % было 11pt
\renewcommand{\cftbeforechapskip}{1em}
\renewcommand{\cftparskip}{-1mm}
\renewcommand{\cftdotsep}{1}
\setcounter{tocdepth}{2} % задать глубину оглавления — до subsection включительно

%%%%%%%%%%%%%%%%%%%%%%%%%% 
% ПОЛЯ
%%%%%%%%%%%%%%%%%%%%%%%%%% 
\usepackage{geometry}
\geometry{left=3cm}
\geometry{right=1.5cm}
\geometry{top=2.4cm}
\geometry{bottom=2.4cm}

%%%%%%%%%%%%%%%%%%%%%%%%%% 
% СПИСКИ
%%%%%%%%%%%%%%%%%%%%%%%%%% 
\usepackage{enumitem}
\makeatletter
\AddEnumerateCounter{\asbuk}{\@asbuk}{м)}
\makeatother
\setlist{nolistsep}
\renewcommand{\labelitemi}{-}
\renewcommand{\labelenumi}{\asbuk{enumi})}
\renewcommand{\labelenumii}{\arabic{enumii})}

%%%%%%%%%%%%%%%%%%%%%%%%%% 
% СПИСОК ЛИТЕРАТУРЫ
%%%%%%%%%%%%%%%%%%%%%%%%%%
%\usepackage{gost}
\usepackage[square,numbers,sort&compress]{natbib}
\renewcommand{\bibnumfmt}[1]{#1.\hfill} % нумерация источников в самом списке — через точку
\renewcommand{\bibsection}{\likechapter{Список використаних джерел}} % заголовок специального раздела
\setlength{\bibsep}{0pt}
\bibliographystyle{ugost2003}

%%%%%%%%%%%%%%%%%%%%%%%%%% 
% СЧЕТЧИКИ
%%%%%%%%%%%%%%%%%%%%%%%%%% 
\usepackage{lastpage}
\newcounter{totfigures}
\newcounter{tottables}
\makeatletter
\AtEndDocument{%
  \addtocounter{totfigures}{\value{figure}}%
  \addtocounter{tottables}{\value{table}}%
  \immediate\write\@mainaux{%
    \string\gdef\string\totfig{\number\value{totfigures}}%
    \string\gdef\string\tottab{\number\value{tottables}}%    
  }%
}
\makeatother

\usepackage{etoolbox}
\pretocmd{\chapter}{\addtocounter{totfigures}{\value{figure}}}{}{}
\pretocmd{\chapter}{\addtocounter{tottables}{\value{table}}}{}{}
%\pretocmd{\likechapter}{\par}

%%%%%%%%%%%%%%%%%%%%%%%%%%
% ДОКУМЕНТ
%%%%%%%%%%%%%%%%%%%%%%%%%% 
\begin{document}
% \begin{titlepage}
  \begin{center}
    \textsc{МІНІСТЕРСТВО ОСВІТИ І НАУКИ УКРАЇНИ \\
      НАЦІОНАЛЬНИЙ ТЕХНІЧНИЙ УНІВЕРСИТЕТ УКРАЇНИ \\}
      <<КИЇВСЬКИЙ~ПОЛІТЕХНІЧНИЙ~ІНСТИТУТ ІМЕНІ~ІГОРЯ~СІКОРСЬКОГО>>\\
      Інженерно-фізичний факультет\\
      Кафедра фізики металів
    \vfill
    \textbf{КУРСОВА РОБОТА\\[3mm]}
      з курсу\\
      <<Електронна мікроскопія 1>>\\[3mm]
      на тему \\
<<Нові застосування трансмісійної електронної мікроскопії>>
      \\[20mm]
  \end{center}
  \hfill
  \begin{minipage}{.5\textwidth}
    {\bf Студента} 4 курсу групи ФМ-31 \\%[2mm]
    напряму підготовки 6.050403\\
    <<Інженерне матеріалознавство>> \\
    Богомаза Р.Д.\\[3mm]

    {\bf Керівник} Котенко І.Є.\\[2mm]
    Національна оцінка: \underline{~~~~~~~~~~~~~~~~~~~~~~~} \\
    Кількість балів: \underline{~~~~} Оцінка: ECTS \underline{~~~~}
  \end{minipage}
  \vfill
  \begin{center}
    Київ 2016
  \end{center}
\end{titlepage}

\tableofcontents
\setcounter{page}{2}
\likechapter{ВСТУП}

Більшість поверхонь сучасних машин піддаються зносу. Як приклад можна навести, зуби шестерень редуктора, направляючі і повзуни, шийки колінчатих валів двигунів, внутрішня поверхня циліндрів двигунів внутрішнього згоряння і тому подібне. Тому до поверхонь деталей можуть висувати ряд вимог, які задають певні механічні властивості: міцність, зносостійкість, корозійна стійкість, жаростійкість, припрацьованість. Існують різні способи поверхневого зміцнення деталей, підвищення їх корозійної стійкості і зниження або збільшення тертя сполучених поверхонь. До них відносяться: поверхнева термічна обробка, легування поверхні деталі наплавленням сплавів, що відповідають необхідним вимогам, гальванічне нанесення на поверхню деталі антикорозійного покриття та інші.

Для зміцнення та нанесення захисних поверхонь вигідно використовувати електрофізичні методи обробки матеріалів, сутність яких полягає на використанні концентрованих потоків енергії. Одним із таких методів є електроіскрове легування (ЕІЛ) металевих поверхонь [1]. Його основними перевагами є економічність, екологічність та простота відтворення. Також цей метод дає широкий спектр можливих варіантів проведення експерименту в залежності від обраних матеріалів та відповідно отриманих властивостей поверхні. Саме по цим критеріям і обраний метод досліження на виробничій практиці.

В даному звіті розглянуті фізичні основи та відомі моделі проходження фізичного процесу електроіскрового легування. Відповідно у роботі наведені методика, результати експериментального дослідження отримання зносостійких поверхонь, висновки на основі літературного огляду та отриманих даних.

\chapter{ЛІТЕРАТУРНИЙ ОГЛЯД}
\section{Загальні відомості}
Метод електроіскрового легування був розроблений Б.Р. Лазаренко та Н.І. Лазаренко в 50-х роках минулого століття. До основних його особливостей варто віднести локальну обробку поверхні — легування  можна проводити в чітко вказаних місцях радіусом у долі міліметрів і більше, без потреби захисту решти поверхні матеріалу; високу міцність зчеплення нанесеного матеріалу з основою; відсутність нагріву деталі в процесі обробки; можливість використання для обробки як чистих металів так їх сплавів, металокерамічних композицій, тугоплавких з'єднань; дифузійне насичення поверхні без зміни розміру основи; відсутність необхідності спеціальної обробки деталі перед легуванням. Технологія доволі проста, необхідне обладнання малогабаритне, надійне та не має особливих проблем з транспортацією \cite{hitlevich-1985}.
Електроіскрове легування використовується для \cite{hiltevich-1985}: % ОСВІЖИТИ ДЖЕРЕЛО ТА ІНФОРМАЦІЮ

\begin{enumerate}
\item збільшення міцності, корозійної, зносо- та жаростійкості;
\item зниження можливості схоплювання деталей при терті;
\item відновлення розмірів інструмента, деталей машин та механізмів;
\item зміни електричних властивостей контактуючих елементів та емісійних властивостей поверхні;
\item надання поверхні необхідних хімічних з'єднань;
\item створення на робочій поверхні шару з необхідною шорсткістю;
\item нанесення радіоактивних ізотопів;
\item застосування декоративному мистецтві. 
\end{enumerate}
Розрізняють два напрямки в електроіскровому легуванні: чистове — коли на поверхні формують тонкий шар до 0,12 мм і грубе — товщина шару може досягати 0,2 мм. На практиці застосовують переважно перший варіант.

\begin{figure}[h!]
  \centering
  \includegraphics[width=\textwidth]{images/test.png}
  \caption{Тестовий тест рисунок кек один два три чотири вісім дев'ять ЕІЛ просто легко і доступно\label{fig:label} }
\end{figure}

Один.

Два.

\pagebreak
\input{conclusions.tex}
\newpage
\bibliography{lit/my}
%\newpage
%\input{app.tex}

\end{document}
