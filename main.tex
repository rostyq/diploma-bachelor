\chapter{ЛІТЕРАТУРНИЙ ОГЛЯД}
\section{Загальні відомості}
Метод електроіскрового легування був розроблений Б.Р. Лазаренко та Н.І. Лазаренко в 50-х роках минулого століття. До основних його особливостей варто віднести локальну обробку поверхні — легування  можна проводити в чітко вказаних місцях радіусом у долі міліметрів і більше, без потреби захисту решти поверхні матеріалу; високу міцність зчеплення нанесеного матеріалу з основою; відсутність нагріву деталі в процесі обробки; можливість використання для обробки як чистих металів так їх сплавів, металокерамічних композицій, тугоплавких з'єднань; дифузійне насичення поверхні без зміни розміру основи; відсутність необхідності спеціальної обробки деталі перед легуванням. Технологія доволі проста, необхідне обладнання малогабаритне, надійне та не має особливих проблем з транспортацією \cite{hitlevich-1985}.
Електроіскрове легування використовується для \cite{hiltevich-1985}: % ОСВІЖИТИ ДЖЕРЕЛО ТА ІНФОРМАЦІЮ

\begin{enumerate}
\item збільшення міцності, корозійної, зносо- та жаростійкості;
\item зниження можливості схоплювання деталей при терті;
\item відновлення розмірів інструмента, деталей машин та механізмів;
\item зміни електричних властивостей контактуючих елементів та емісійних властивостей поверхні;
\item надання поверхні необхідних хімічних з'єднань;
\item створення на робочій поверхні шару з необхідною шорсткістю;
\item нанесення радіоактивних ізотопів;
\item застосування декоративному мистецтві. 
\end{enumerate}
Розрізняють два напрямки в електроіскровому легуванні: чистове — коли на поверхні формують тонкий шар до 0,12 мм і грубе — товщина шару може досягати 0,2 мм. На практиці застосовують переважно перший варіант.

\begin{figure}[h!]
  \centering
  \includegraphics[width=\textwidth]{images/test.png}
  \caption{Тестовий тест рисунок кек один два три чотири вісім дев'ять ЕІЛ просто легко і доступно\label{fig:label} }
\end{figure}

Один.

Два.
